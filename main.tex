\documentclass[a4paper, titlepage]{article}

\usepackage[utf8]{inputenc}
\usepackage[english]{babel}
\usepackage[backend=bibtex, style=alphabetic]{biblatex}
\usepackage{amsmath}
\usepackage{amssymb}
\usepackage{commath}
\usepackage{csquotes}
\usepackage{epigraph}
\usepackage{siunitx}
\usepackage{hyperref}
\usepackage[lastexercise]{exercise} % The answer will be linked to the last exercise.
\usepackage{titlesec}
\usepackage{indentfirst}

\addbibresource{biblio.bib}

\newcommand{\sectionbreak}{\clearpage}

\title{Solved Calculus Exercises}
\author{Bernardo Sulzbach}
\date{May 2016}

\begin{document}

\maketitle

\tableofcontents

\section{Introduction}

\setlength{\epigraphwidth}{3in}
\epigraph{There exists no separation between gods and men; one blends softly
casual into the other.}{Frank Herbert}

\subsection{About this document}

This is a collection of calculus exercises with their solutions. These pages do
not aim to teach calculus from the ground up, but to present complete solutions
to several problems one may come across.

Some of these problems were extracted from popular calculus books.
Where this was the case, the source is mentioned.

Problems whose source is not acknowledged were either conceived by
the author or are freely published somewhere else.

The source code of this book is made available on its
\href{https://github.com/mafagafogigante/solved-calculus-exercises}{GitHub
repository} and the original author may be reached at
\href{mailto:mafagafogigante@gmail.com}{mafagafogigante@gmail.com}.

\subsection{On typographical style}

The main author of this document agrees with the \textit{Imprimerie nationale},
about always indenting after titles. The reader might have already noticed this
by now.

\subsection{Regarding collaboration}

This project is free and open source. Therefore, suggestions and patches are
accepted and welcome. If you find mathematical errors, misspelled words, or
formatting issues, please contact me so I can improve the quality of the
material. Exercise submissions - with or without solutions - are also welcome.
Try, however, not to submit an exercise which is identical to one this document
already has except for some constant factors.

\section{Differentiation}

\subsection{Optimization problems}

\begin{Exercise}
A rectangular field is to be bounded by a fence on three sides
and by a straight stream on the fourth side.
Find the dimensions of the field with maximum area that can be enclosed
using 1000 ft of fence.
\cite{anton-bivens-davis}
\end{Exercise}

\begin{Answer}
\[A = x \left( 1000 - 2x \right)\]

\begin{align*}
    \frac{d}{dx} A &= 0 \\
    \frac{d}{dx} A &= \frac{d}{dx} x \left( 1000 - 2x \right) \\
    &= \frac{d}{dx} -2x^2 + 1000x \\
    &= -4x + 1000
    \iff x = 250
\end{align*}

The best rectangle should have a side parallel to the stream of 500 ft
and two sides of 250 ft.
\end{Answer}

\begin{Exercise}
Find the dimensions of the rectangle with maximum area
that can be inscribed in a circle of radius 10.
\cite{anton-bivens-davis}
\end{Exercise}

\begin{Answer}
Let \(a\) and \(b\) are the sides of the rectangle. Then \(A = ab\) and
\[r^2 = 100 = \left( \frac{a}{2} \right)^2 + \left( \frac{b}{2} \right)^2\]
Therefore,
\[b^2 = 400 - a^2\]
Therefore,
\[A = ab = a \sqrt{400 - a^2}\]
\begin{align*}
    \od{}{a} A &= \od{}{a} a \sqrt{400 - a^2} \\
    &= \left( \od{}{a} a \right) \left( \sqrt{400 - a^2} \right)
      + \left( a \right) \left( \od{}{a} \sqrt{400 - a^2} \right) \\
    &= \sqrt{400 - a^2} + a \left(\frac{-a}{\sqrt{400 - a^2}} \right) \\
    &= \sqrt{400 - a^2} + \frac{-a^2}{\sqrt{400 - a^2}} \\
    &= \frac{400 - a^2 - a^2}{\sqrt{400 - a^2}} \\
    &= \frac{400 - 2a^2}{\sqrt{400 - a^2}} = 0 \\
    &\therefore 2a^2 = 400 \rightarrow a = \sqrt{200} \\
\end{align*}

\[b = \sqrt{400 - a^2} = \sqrt{200}\]
\end{Answer}

\begin{Exercise}
A box with a square base is taller than it is wide. In order
to send the box through the U.S. mail, the height of the box
and the perimeter of the base can sum to no more than 108
centimeters. What is the maximum volume for such a box?
\cite{anton-bivens-davis}
\end{Exercise}

\begin{Answer}

\[V = a^2 b, b > a\]
\[b + 4a \le \SI{108}{\centi\meter}\]

Therefore,
\[b \le \SI{108}{\centi\meter} - 4a\]

As using any value of \(b\) such that \(b < \SI{108}{\centi\meter} - 4a\)
will be suboptimal (because \(b\) is positive and \(V\) is proportional to \(b\)),
we take \(b = \SI{108}{\centi\meter} - 4a\).

\begin{align*}
  \od{}{a} a^2 b &= 0 \\
  \od{}{a} a^2 b &= \od{}{a} a^2 \left( \SI{108}{\centi\meter} - 4a \right) \\
                 &= -12 a^2 + \left( \SI{216}{\centi\meter} \right) a \\
                 &\therefore 12 a^2 = \left( \SI{216}{\centi\meter} \right) a \\
                 &\therefore 12 a = \SI{216}{\centi\meter} \\
                 &\therefore a = \SI{18}{\centi\meter}
\end{align*}
\[V = a^2 b = \left( \SI{18}{\centi\meter} \right)^2 \left( \SI{36}{\centi\meter} \right)
            = \SI{11664}{\cubic\centi\meter}\]
\end{Answer}

\begin{Exercise}
Find the height and radius of the right circular cone with
least volume that can be circumscribed about a sphere of
radius \(R\).
\cite{anton-bivens-davis}
\end{Exercise}

\begin{Answer}

\[V_{cone} = \frac{r^2 h \pi}{3}\]

Let \(r\) be the radius of the base of the cone and \(h\) be its height.

We make use of similarity of triangles to express \(r\) in terms of \(h\).

\begin{align*}
  \frac{R}{h - R} &= \frac{r}{\sqrt{r^2 + h^2}} \\
   R \sqrt{r^2 + h^2} &= r\left(h - R\right) \\
   R^2 \left( r^2 + h^2 \right) &= r^2 \left(h^2 - 2 h R + R^2\right) \\
   r^2 \left(h^2 - 2 h R\right) &= R^2 h^2 \\
   r^2 &= \frac{R^2 h}{h - 2 R}
\end{align*}

\[V_{cone} = \frac{r^2 h \pi}{3} = \frac{R^2 h^2 \pi}{3 h - 6 R} \]

\[\dod{}{h} \frac{R^2 h^2 \pi}{3 h - 6 R} =
  \frac{\left(2 R^2 h \pi \right) \left(3 h - 6 R \right) - 3 \left( R^2 h^2 \pi \right)}
       {\left( 3 h - 6 R \right)^2}\]

\[\frac{\left(2 R^2 h \pi \right) \left(3 h - 6 R \right) - 3 \left( R^2 h^2 \pi \right)}
       {\left( 3 h - 6 R \right)^2} = 0\]
\[\left(2 R^2 h \pi \right) \left(3 h - 6 R \right) = 3 \left( R^2 h^2 \pi \right)\]
\[2h - 4R = h\]
\[h = 4R\]
\[r = \sqrt{\frac{R^2 h}{h - 2R}} = \sqrt{\frac{4R^3}{4R - 2R}} = \sqrt{2} R\]
\end{Answer}

\begin{Exercise}
Find the height and radius of the largest right circular cone that can be inscribed in a sphere of radius \(R\).
\end{Exercise}
\begin{Answer}

\[V_{cone} = \frac{r^2 h \pi}{3}\]
\[\left(h - R \right)^2 + r^2 = R^2\]
\[h^2 - 2hR + r^2 = 0\]
\[h \left( 2R - h \right) = r^2\]

\[V_{cone} = \frac{h \left( 2R - h \right) h \pi}{3}\]
\[V_{cone} = \frac{\left( -h^3 + 2h^2R \right) \pi}{3}\]

\begin{align*}
    \dod{}{h} V_{cone} &= \dod{}{h} \frac{\left( -h^3 + 2h^2R \right) \pi}{3} \\
     &= \frac{\left( -3h^2 + 4hR \right) \pi}{3} \\
\end{align*}


\[\frac{\left( -3h^2 + 4hR \right) \pi}{3} = 0\]
\[-3h^2 + 4hR = 0\]
\[h = \frac{4R}{3}\]
\[r = \sqrt{h \left(2R - h\right)}
    = \sqrt{\frac{4R}{3} \left(\frac{2R}{3}\right)}
    = 2 \sqrt{\frac{2}{3}} R\]
\end{Answer}

\begin{Exercise}
The lower edge of a painting, 10 ft in height, is 2 ft above
an observer’s eye level. Assuming that the best view is obtained
when the angle subtended at the observer's eye by the painting
is maximum, how far from the wall should the observer stand?
\cite{anton-bivens-davis}
\end{Exercise}

\begin{Answer}

Let \(\theta\) be the angle subtended at the observer's eye
and \(x\) be the distance between the observer and the wall.

\[\theta = \arctan \left( \frac{10 + 2}{x} \right) - \arctan \left( \frac{2}{x} \right)\]

We first find the derivative of \(\theta\) with respect to \(x\).

\begin{align*}
  \dod{\theta}{x} &= \dod{}{x} \left( \arctan \left( \frac{10 + 2}{x} \right) - \arctan \left( \frac{2}{x} \right) \right) \\
   &= \dod{}{x} \left( \arctan \left( \frac{10 + 2}{x} \right) \right)
    - \dod{}{x} \left( \arctan \left( \frac{2}{x} \right) \right) \\
   &= \frac{-12}{144 + x^2} - \frac{-2}{4 + x^2} \\
   &= \frac{-12 \left( 4 + x^2 \right) + 2 \left( 144 + x^2 \right)}
           {\left( 144 + x^2 \right) \left( 4 + x^2 \right)} \\
\end{align*}

Then we find out when the derivative is zero.

\begin{align*}
  \frac{-12 \left( 4 + x^2 \right) + 2 \left( 144 + x^2 \right)}
       {\left( 144 + x^2 \right) \left( 4 + x^2 \right)} &= 0 \\
  \left( 144 + x^2 \right) - 6 \left( 4 + x^2 \right) &= 0 \\
  120 - 5 x^2 &= 0 \\
  x &= \pm \sqrt{24}
\end{align*}

As \(x\) must be nonnegative, we discard \(- \sqrt{24}\).

Lastly, we must see what is the sign of the derivative before and after the point where it
equals zero for positive values of \(x\). I chose to test 4 and 5.

\begin{align*}
  \dod{\theta}{x}\Bigr|_{\substack{\\x = 4}} &= \frac{1}{40} \\
  \dod{\theta}{x}\Bigr|_{\substack{\\x = 5}} &= - \frac{10}{4901}
\end{align*}

As there is only one positive value of \(x\) such that \(\dod{\theta}{x}\) is zero
and it is positive to the left and negative to the right, this point is an absolute maximum
and the answer is 24 feet.

\end{Answer}

\section{Integration}

\subsection{Introduction}

Several exercises in this section are solved with formulas found in most
integration tables. This book does not present nor proves these formulas as this
is out of its scope.

\subsection{The power rule}

\begin{Exercise}
\[\int \left( x^{-3} - 3x^{1/4} + 8x^2 \right) dx\]
\cite{anton-bivens-davis}
\end{Exercise}

\begin{Answer}
\begin{align*}
\int \left( x^{-3} - 3x^{1/4} + 8x^2 \right) dx
&= \int x^{-3} dx - \int 3x^{1/4} dx + \int 8x^2 dx \\
&= - \frac{1}{2}x^{-2} - \frac{12}{5}x^{5/4} + \frac{8}{3}x^3 + C
\end{align*}
\end{Answer}

\begin{Exercise}
\[\int \left( \frac{2}{x} + 3e^x \right) dx\]
\cite{anton-bivens-davis}
\end{Exercise}

\begin{Answer}
\begin{align*}
\int \left( \frac{2}{x} + 3e^x \right) dx
&= \int 2x^{-1} dx + \int 3e^x dx \\
&= 2 \int x^{-1} dx + 3 \int e^x dx \\
&= 2 \ln \abs{x} + 3 e^x + C
\end{align*}
\end{Answer}

\begin{Exercise}
\[\int \left( 3 \sin x - 2 \sec^2 x \right) dx\]
\cite{anton-bivens-davis}
\end{Exercise}

\begin{Answer}
\begin{align*}
\int \left( 3 \sin x - 2 \sec^2 x \right) dx
&= 3 \int \sin x dx - 2 \int \sec^2 x dx \\
&= - 3 \cos x - 2 \tan x + C
\end{align*}
\end{Answer}

\begin{Exercise}
    \[\int \left( \sec x \left( \sec x + \tan x \right) \right) dx\]
\cite{anton-bivens-davis}
\end{Exercise}

\begin{Answer}
\begin{align*}
    \int \left( \sec x \left( \sec x + \tan x \right) \right) dx \\
    &= \int \left( \sec^2 x \right) dx + \int \left( \sec x \tan x \right) dx \\
    &= \tan x + \sec x + C
\end{align*}
\end{Answer}

\begin{Exercise}
    \[\int \left( \frac{1}{2 \sqrt{1 - x^2}} - \frac{3}{1 + x^2} \right) dx\]
\cite{anton-bivens-davis}
\end{Exercise}

\begin{Answer}
\begin{align*}
    \int \left( \frac{1}{2 \sqrt{1 - x^2}} - \frac{3}{1 + x^2} \right) dx
    &= \frac{1}{2} \int \frac{1}{\sqrt{1 - x^2}} dx
    - 3 \int \frac{1}{1 + x^2} dx \\
    &= \frac{1}{2} \sin^{-1} x - 3 \tan^{-1} x + C
\end{align*}
\end{Answer}

\begin{Exercise}
\[\int \left( \frac{1}{1 + \sin x} \right) dx\]
\cite{anton-bivens-davis}
\end{Exercise}

\begin{Answer}
\begin{align*}
    \int \left( \frac{1}{1 + \sin x} \right) dx
    &= \int \left( \frac{1}{1 + \sin x} \frac{1 - \sin x}{1 - \sin x} \right) dx \\
    &= \int \left( \frac{1 - \sin x}{1 - \sin^2 x} \right) dx \\
    &= \int \left( \frac{1 - \sin x}{\cos^2 x} \right) dx \\
    &= \int \left( \frac{1}{\cos^2 x} - \frac{\sin x}{\cos^2 x} \right) dx \\
    &= \int \left( \frac{1}{\cos^2 x} \right) dx - \int \left( \frac{\sin x}{\cos^2 x} \right) dx \\
    &= \int \left( \sec^2 x \right) dx - \int \left( \sec x \tan x \right) dx \\
    &= \tan x - \sec x + C
\end{align*}
\end{Answer}

\subsection{Initial value problems}

\begin{Exercise}
    \begin{align*}
        \dod{y}{x} &= \sqrt[3]{x} \\
        y \left( 1 \right) &= 2
    \end{align*}
    \cite{anton-bivens-davis}
\end{Exercise}

\begin{Answer}
    We start by integrating the derivative we are given to find the general solution to this initial value problem (IVP).
    \[\int \dod{y}{x} dx = \frac{3}{4}x^{4/3} + C\]

    Then, we solve for \(C\) by using the second equation we are given.
    \[y \left( 1 \right) = \frac{3}{4} + C = 2 \iff C = \frac{5}{4}\]

    This gives us the actual solution for this IVP, which is
    \[y \left( x \right) = \frac{3}{4}x^{4/3} + \frac{5}{4}\]
\end{Answer}

\begin{Exercise}
    \begin{align*}
        \dod{y}{x} &= \frac{x + 1}{\sqrt{x}} \\
        y \left( 1 \right) &= 0
    \end{align*}
    \cite{anton-bivens-davis}
\end{Exercise}

\begin{Answer}
    \begin{align*}
	    \int \dod{y}{x} dx &= \int \left( x^{1/2} + x^{-1/2} \right) dx \\
	    &= \frac{2}{3}x^{3/2} + 2x^{1/2} + C
    \end{align*}

    \[y \left( 1 \right) = \frac{2}{3} + 2 + C = 0 \iff -\frac{8}{3} \]

    Therefore, the actual solution for this IVP is
    \[y \left( x \right) = \frac{2}{3}x^{3/2} + 2x^{1/2} - \frac{8}{3}\]
\end{Answer}

\subsection{Integration by parts}

\begin{Exercise}
\[\int e^x \cos x \mathop{dx}\]
\end{Exercise}

\begin{Answer}

We use integration by parts to derive that

\[\int e^x \cos x \mathop{dx}
  = e^x \sin x - \int e^x \sin x \mathop{dx}\]

Another step of integration by parts shows us that

    \[\int e^x \sin x \mathop{dx} = - e^x \cos x + \int e^x \cos x \mathop{dx}\]

And therefore that

\begin{align*}
\int e^x \cos x \mathop{dx} &= e^x \sin x + e^x \cos x - \int e^x \cos x \mathop{dx} \\
 &= \frac{e^x \left(\sin x + \cos x \right)}{2} + C
\end{align*}
\end{Answer}

\printbibliography

\end{document}
